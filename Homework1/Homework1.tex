\documentclass{article}
\usepackage{amsmath}
\author{Ben Schlegel
}
\title{Homework 1}
\date{\today}
\begin{document}
\maketitle
\section{} (Please grade this one)
\noindent \textbf{If \emph{r} is rational ($r \neq 0$) and \emph{x} is irrational, prove that $r + x$ and $rx$ are irrational.}\newline
Say $r = \frac{a}{b}$, and that $r + x$ is rational. Therefore:
\begin{align*}
\frac{a}{b} + x = \frac{m}{n}\\
x = \frac{m}{n} - \frac{a}{b}\\
x = \frac{m - a }{n - b}
\end{align*}
This would mean x is irrational, which is a contradiction\\
Say $r = \frac{a}{b}$, and that $rx$ is rational. Therefore:
\begin{align*}
\frac{ax}{b} = \frac{m}{n}\\
x = \frac{bm}{an} 
\end{align*}
Again, this means x is rational

\section{}
\noindent \textbf{Prove that there is no rational number whose square is 12}\newline 
Assume $\sqrt{12}$ is rational. Then there exists integers \emph{a} and \emph{b} share no common factors and $\frac{a}{b} = \sqrt{12}$
\begin{align*}
\frac{a^2}{b^2} = 12
\end{align*}
Because \emph{a} and \emph{b} share no common factors, $b^2$ must be 1 in order for the fraction to produce an integer. Therefore
\begin{align*}
a^2 = 12
\end{align*}
There are no integers whose square equal 12, so we have reached a contradiciton.
\section{}
\noindent \textbf{Suppose that $z = a + bi$, $w = u + iv$, and} \begin{center} $a = (\frac{\mid w \mid + u}{2})^{1/2}$, $b = (\frac{\mid w \mid - u}{2})^{1/2}$ \end{center}
\textbf{Prove that $z^2 = w$ if $v \ge 0$ and that $\overline{z}^2 = w$ if $v \le 0$. Conclude that every complex number (with one exception!) has two complex square roots.}
\begin{align*}
z^2 = a^2 + 2abi - b^2\\
= (\frac{ |w| + u}{2}) + 2abi - (\frac{|w| - u}{2})\\
 = u + 2abi\\
 = u + 2\sqrt{\frac{1}{4}(|w|^2 - u^2)}i\\
 = u + \sqrt{|w|^2 - u^2}i\\
 = u + \sqrt{u^2 + v^2 - u^2}i\\
 z^2 = u + vi 
\end{align*}

\section{}
\noindent \textbf{If $z_1$,..., $z_n$ are complex, prove that:} \begin{center} $|z_1 + z_2 + \cdots + z_n| \le |z_1| + |z_2| + \cdots + |z_n| $\end{center} 
\begin{align*}
\sqrt{(a_1 + b_1)^2+(a_2 + b_2)^2} \le \sqrt{(a_1 + b_1)^2} + \sqrt{(a_2 + b_2)^2}\\
(a_1 + b_1)^2+(a_2 + b_2)^2 \le a_1^2 + a_2^2 + b_1^2 + b2^2 + 2\sqrt{(a_1^2 + a_2^2)(b_1^2 + b_2^2)}\\
a_1^2 + 2a_1 b_1 + b_1^2 + a_2^2 + 2a_2 b_2 + b_2^2 \le a_1^2 + a_2^2 + b_1^2 + b2^2 + 2\sqrt{(a_1^2 + a_2^2)(b_1^2 + b_2^2)}\\
a_1b_1 + a_2b_2 \le \sqrt{(a_1^2 + a_2^2)(b_1^2 + b_2^2)}\\
(a_1b_1 + a_2b_2)^2 \le (a_1^2 + a_2^2)(b_1^2 + b_2^2)\\
a_1^2b_1^2 + 2a_1b_1a_2b_2 + a_2^2b_2^2 \le a_1^2b_1^2 + a_2^2b_2^2 + a_1^2b_2^2 + a_2^2b_1^2\\
2a_1b_1a_2b_2 \le a_1^2b_2^2 + a_2^2b_1^2\\
0 \le a_1^2b_2^2 - 2a_1b_1a_2b_2 +  a_2^2b_1^2\\
0 \le (a_1b_2 - a_2b_1)^2
\end{align*}
This will always be true, and since the above inequality can be simplified down into this, that will also always be true


\section{}
\noindent \textbf{If \emph{x,y} are complex, prove that:} \begin{center} $||x| - |y|| \le |x-y| $ \end{center}
Let $x = a + bi$ and $y = c + di$ (subscripts were tedious and not very neat last time):
\begin{align*}
|\sqrt{a^2 + b^2} - \sqrt{c^2 + d^2}| \le \sqrt{(a-c)^2 + (b - d)^2}\\
a^2 + b^2 -2\sqrt{(a^2+b^2)(c^2 + d^2)} + c^2 + d^2 \le (a-c)^2 + (b - d)^2\\
a^2 + b^2 -2\sqrt{(a^2+b^2)(c^2 + d^2)} + c^2 + d^2 \le a^2 - 2ac +c^2 + b^2 -2bd + d^2\\
\sqrt{(a^2+b^2)(c^2 + d^2)} \le ac + bd\\
(a^2+b^2)(c^2 + d^2) \le a^2c^2 + 2abcd + b^2d^2\\
a^2c^2 + a^2d^2 + b^2c^2 + b^2d^2 \le  a^2c^2 + 2abcd + b^2d^2\\
a^2d^2 +  b^2c^2 \le 2abcd\\
0 \le (ad - bc)^2
\end{align*}

\section{}
\noindent \textbf{If z is a complex number such that $|z| = 1$, that is, such that $z\overline{z} = 1$, compute} \begin{center}$|1+z|^2 + |1-z|^2$ \end{center} 
Say $z = a + bi$
\begin{align*}
|(a + 1) + bi|^2 + |(a-1) + bi|\\
(a+1)^2 + b^2 + (1-a)^2 + b^2\\
a^2 + 2a + 1 + b^2 + 1 - 2a + a^2 + b^2\\
2a^2 + 2b^2 + 2\\
2(a^2 + b^2) + 2\\
2|z| + 2 = 4
\end{align*}

\section{} (Please grade this one)
\noindent \textbf{Under what conditions does equality hold in the Schawrz inequality}
The Schwarz inequality is equal when the two vectors are pointed in the same direction/one is a scalar multiple of the other or \emph{vis versa}
\end{document} 